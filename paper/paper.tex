% !TeX root = paper.tex
% !TeX TXS-program:compile = txs:///pdflatex/[--shell-escape]
% https://orcid.org/0000-0003-4586-8500
\documentclass{article}
\usepackage{blindtext}

%%% FONTS %%%
\usepackage[T1]{fontenc}
\usepackage[utf8]{inputenc}
\usepackage[english]{babel}
%\usepackage{tgpagella} % set the document font to TeX Gyre Pagella
%\usepackage{tgbonum} % set the document font to TeX Gyre Bonum
%\usepackage{fontawesome5} % The Creative Commons icons

\usepackage{glossaries}

%%% DRAFT watermark %%%
\usepackage{draftwatermark}
\SetWatermarkText{DRAFT}
\SetWatermarkScale{5.1}
\SetWatermarkLightness{0.8}

\usepackage{xcolor} % \textcolor{red}{text} will be red for notes
\definecolor{lightgray}{gray}{0.6}
\definecolor{medgray}{gray}{0.4}

\usepackage{hyperref}
\hypersetup{
	colorlinks=true,
	urlcolor= blue,
	citecolor=blue,
	linkcolor= blue,
	% bookmarks=true,
	% bookmarksopen=false,
}

% for inserting images in line
\usepackage{graphicx}
\graphicspath{ {images} }

% Allows to rewrite the same title in the supplement
\newcommand{\mytitle}{Secure Network Authentication with Kerberos and OpenLDAP}
\usepackage{authblk}
\usepackage{comment} % allows block comments

\usepackage{ragged2e} % use flush and justify for text blocks
\usepackage{csquotes} % use \displayquote{} in the doc

%%% GRAPHICS  AND CODE BLOCKS %%%
\usepackage[listings, minted]{tcolorbox}
\usepackage{xcolor,colortbl}
\definecolor{myblue}{RGB}{0,163,243}
\definecolor{mygrey}{RGB}{128,128,128}
\definecolor{whitesmoke}{RGB}{245,245,245}
\newtcolorbox[auto counter, number within=section]{mybox}[2][]{
	colbacktitle=mygrey,
	colback=whitesmoke,
	title={#2},
	fonttitle=\ttfamily\small,
	fontupper=\sffamily\tiny,
	halign=flush left,
	rounded corners
}

%\usepackage[verbose]{wrapfig} % wrap text around image

\definecolor{myblue}{HTML}{4285F4} % color table cells

\usepackage[tablegrid]{vhistory} % version history package

% Headers and footers
\usepackage{fancyhdr}
\pagestyle{fancy}
\fancyhf{}
\lhead{Secure Network Authentication with Kerberos and OpenLDAP}
\lfoot{\tiny{September 29, 2021}}
\rfoot{\tiny{version: \vhCurrentVersion}}

% Code to add paragraph numbers and titles
\newif\ifptitle
\newif\ifpnumber
\newcounter{para}[section] % section x.y that resets in each new section
\newcommand\ptitle[1]{\par\refstepcounter{para}
	{\ifpnumber{\noindent\textcolor{lightgray}{\textbf{\thesection.\thepara}}\indent}\fi}
	{\ifptitle{\textbf{[{#1}]}}\fi}}
\ptitletrue  % comment this line to hide paragraph titles
\pnumbertrue  % comment this line to hide paragraph numbers


\begin{document}

% frontmatter: half title, title page, colophon (copyright page), epigraph, toc, preface, acknowledgements
\title{\mytitle}
\author[1,2]{Franklin E. Diaz\\ \texttt\href{emailto: fdiaz@paloaltonetworks.com}{fdiaz@paloaltonetworks.com}}
\affil[1]{Palo Alto Networks}
\affil[2]{Professional Services - Automation}
\begin{titlepage}
	\maketitle
\begin{abstract}
	The goal of this project paper is to detail my use of MIT Kerberos and OpenLDAP as an authentication mechanism for private cloud 
    computing environments. In addition to how the authentication mechanism is installed and configured, this paper will illustrate ways of
	using the mechanism to provision user accounts and show some examples of how the framework might be employed. The intent
	is to create an implementation that can be easily reproduced.
\end{abstract}
\end{titlepage}

\begin{comment}
Source files for this paper are available at: \url{https://github.com/devsecfranklin/paper-template/tree/main/}
\end{comment}


%%% front cover, spline, back cover
% front_cover

\section{\label{sec:intro}Secure Network Authentication with Kerberos and OpenLDAP}
\vspace{2mm}

\justifying
Secure authentication is a primary concern in multi-user networks. Simply put, network administrators are tasked with ensuring 
valid users are able to access network resources, and the rest of the world is not. For this project, MIT Kerberos\cite{krb}\cite{what-is-krb} provides
a secure authentication mechanism, using \href{https://www.openldap.org/}{OpenLDAP} schemas for defining user accounts.

\section{\label{sec:Project}Project Goals}
\vspace{2mm}
\justifying
There are several goals related to this project. These goals overlap and interconnect at times. The overarching goal is to understand the composition,
management, and weaknesses of the items listed here.

\begin{raggedright}
	\begin{enumerate}
		\item Normalize the installation and use of secure authentication methodology using Open Source software.
		\item Move away from SSH keys and complicated ``key management'' schemes.
		\item Provide a working example to make this authentication paradigm easier to reproduce.
	\end{enumerate}
\end{raggedright}
\vspace{2mm}

\justifying
There are websites and documentation that detail installation and some configuration tasks related to this authentication paradigm.
It can be difficult to find practical examples and detailed demonstration of usage and usability. 

\section{\label{sec:environment}The Lab Environment}
\vspace{2mm}
\justifying
This section is a brief description of the ``Private Cloud'' lab environment in which the project was built and deployed. 

\begin{raggedright}
	\begin{enumerate}
		\item Single Board Compute (SBC) devices were chosen for their low power consumption/heat output and silent operation when possible.
		\item Cluster nodes are meant to be added and rebuilt quickly and easily. Use Ansible and other automation whenever possible.
	\end{enumerate}
\end{raggedright}

\vspace{2mm}
\justifying
The core of the Kerberos authentication implementation is the Key Distribution Center (KDC). In this lab configuration, the KDC is running on
an O-droid C1 running Ubuntu.

%\begin{figure}[H]
%cludegraphics[width=12cm]{../images/lab.png}
%\caption{The project environment used for testing.}
%\label{lab}
%end{figure}


\justifying
A Palo Alto Networks VM-220 is used to pass traffic in and out of the cluster. The firewall has also been configured to serve DHCP to clients,
though this service could be placed elsewhere in the network with minimal effort.

\vspace{2mm}
\ptitle{DNS Configuration}
\vspace{2mm}

\justifying
It is possible for clients to locate a KDC through URI records in DNS. While \href{https://web.mit.edu/kerberos/www/krb5-latest/doc/admin/realm_config.html#kdc-discovery}{the MIT Kerberos Documentation}
does provide a good explanation and an example configuration, some changes were needed to make the feature behave correctly with Debian 10 Buster and BIND9 with private/local DNS servers.

\begin{mybox}{\thetcbcounter: Configuring DNS for Kerberos}
	\lstinputlisting{code/krb-dns.txt}
\end{mybox}

\justifying
Validation of the proper DNS configuration can be performed via command line. 

\begin{mybox}{\thetcbcounter: Verifying DNS for Kerberos}
	\lstinputlisting{code/krb-dns-2}
\end{mybox}


\vspace{2mm}
\ptitle{NTP Configuration}
\vspace{2mm}


\justifying
A reliable source of time for the network is critical for this project. An example NTP configuration via \href{https://github.com/devsecfranklin/paper-kerberos-openldap/blob/main/ansible/roles/common/tasks/ntp.yml}{Ansible playbook has been included} for reference.
While it is possible to \href{http://www.vervest.org/htp/}{synchronize time via the HTTP network time protocol}, it is highly discouraged for this authentication methodology.

\section{\label{sec:krb}Kerberos}

\vspace{2mm}
\ptitle{Kerberos Server}
\vspace{2mm}

\justifying
Kerberos KDC servers are configured using the files listed in the following table. Examples are 
provided in the ``ansible/'' folder in source repository of this paper.

\begin{raggedright}
	\begin{enumerate}
		\item /etc/krb5.conf
		\item /etc/krb5kdc/kadm5.acl
		\item /etc/krb5kdc/kdc.conf
		
	\end{enumerate}
\end{raggedright}

\vspace{2mm}
\ptitle{Kerberos Client}
\vspace{2mm}

Client configurations tend to be a bit more complex than configuring the servers simply because there are more touch points.

\vspace{2mm}
\ptitle{Principals and Keytabs}
\vspace{2mm}

\justifying
A Kerberos principal is a user or service upon with certain access related entitlements are bestowed. Administrators use the ``kadmin''
command to \href{https://web.mit.edu/kerberos/krb5-1.19/doc/admin/princ_dns.html}{define the user and service principals}. Consider the following typical 
provisioning session.

\begin{mybox}{\thetcbcounter: Keytab Creation}
	\lstinputlisting{code/kt.txt}
\end{mybox}

The keytabs present on any given host in the network will look similar to the following code block. Note that the ``kadmin'' command is run on the
host where the keytab file should ultimately reside. This allows for the ``klist'' command to successfully show the ``host'' and ``ldap'' principals
in the output.

\begin{mybox}{\thetcbcounter: Viewing Keytabs}
	\lstinputlisting{code/keytabs}
\end{mybox}

\section{\label{sec:openldap}OpenLDAP}

The OpenLDAP portion of the configuration serves as a backend storage for details about user accounts. User accounts are stored as objects with
unique ``distinguished names'' (DNs). 

\vspace{2mm}
\ptitle{Supported SASL Mechanisms}
\vspace{2mm}

\begin{mybox}{\thetcbcounter: Viewing the Support SASL Mechanisms}
	\lstinputlisting{code/ldap-sasl}
\end{mybox}


\justifying
The ``/etc/nsswitch.conf'' file is used to define the type and order of the methods that will be
used for lookups on things like the password, group, etc for a user. One the OpenLDAP server is installed and configured properly, the nsswitch file is updated so the host will perform lookups
against the LDAP database.

\begin{mybox}{\thetcbcounter: The /etc/nsswitch.conf file}
	\lstinputlisting{code/nsswitch.conf}
\end{mybox}

\section{\label{sec:integration}Integrations}

An application that is configured to work with the Kerberos authentication framework is said to be ``Kerberized''.

\vspace{2mm}
\ptitle{Kerberized SSH}
\vspace{2mm}

\justifying
Integration of Secure Shell (SSH) with Kerberos is a common use case. Attackers are actively looking for SSH keys that they can exploit\cite{ssh-key}. SSH
keys and other sensitive credentials are often inadvertently committed to revision control systems by developers. These can be 
\href{https://docs.github.com/en/authentication/keeping-your-account-and-data-secure/removing-sensitive-data-from-a-repository#fully-removing-the-data-from-github}{quite difficult to remove from the history of a repository} and may even necessitate deleting and recreating the repository. Leaving it up to the end user to
opt-in to the SSH key management initiatives outlined by IT or Security arm of the organization brings risk. Centralization of user and password management
means reducing this risk.
\vspace{2mm}

\justifying
Updates to the ``/etc/ssh/ssh\_config'' and ``/etc/ssh/sshd\_config'' files are made to allow Kerberos interactions with SSH on the incoming and outgoing sessions respectively. Modification of the user configuration can be done by the user or enforced via Ansible playbook to ensure compliance. Settings in the aforementioned server level configuration may be overridden if permitted.
\vspace{2mm}

\begin{mybox}{\thetcbcounter: User's \$\{HOME\}/.ssh/config file}
	\lstinputlisting{code/ssh-config}
\end{mybox}
\vspace{2mm}

\justifying
Once a user has successfully authenticated to the KDC using the kinit command, hosts can be configured to allow logins with no password.

\vspace{2mm}
\begin{mybox}{\thetcbcounter: Obtaining a ticket from the KDC.}
	\lstinputlisting{code/kinit-session}
\end{mybox}
\vspace{2mm}

\justifying
The project support \href{https://www.rfc-editor.org/rfc/rfc4462.html#page-13}{authentication using GSS-API user authentication} as well as \href{https://www.rfc-editor.org/rfc/rfc4462.html#page-19}{authentication using GSS-API key exchange} as described in RFC 4462\cite{rfc4462}.

\justifying
The next figure shows the debug from a typical SSH session with the ``gssapi-with-mic'' authentication method enabled. Note that the ``-K'' flag is typically not required since
we've specified our setup in ``\$\{HOME\}/.ssh/config''.

\vspace{2mm}
\begin{mybox}{\thetcbcounter: SSH Connection with Kerberos}
	\lstinputlisting{code/ssh-session}
\end{mybox}

\justifying
A similar method could be used to enable/validate the GSS-API key exchange method as described in the reference document\cite{rfc4462}.

\vspace{2mm}
\ptitle{Ansible Integration}
\vspace{2mm}

\justifying
\href{https://www.ansible.com/overview/how-ansible-works}{Ansible playbooks} can be used to deploy and manage the entire authentication ecosystem,
be it in public/private networks or cloud environments. The \href{https://docs.ansible.com/ansible/2.3/intro_inventory.html#hosts-and-groups}{Ansible inventory is used to group and manage your hosts} in ways that make sense for your 
network.

\justifying
Once the KDC and OpenLDAP servers are running, you can authenticate your Ansible plays against the KDC. Note the specification of the Kerberos Realm in the Ansible hosts file. The realm is appended to each host in all capital letters.

\begin{mybox}{\thetcbcounter: Authorizing Ansible via Kerberos}
	\lstinputlisting{code/ansible-auth.txt}
\end{mybox}

\vspace{2mm}
\ptitle{Pan OS Integration}
\vspace{2mm}

\justifying
Once the KDC is up and the user has been provisioned with an LDAP account, \href{https://docs.paloaltonetworks.com/pan-os/10-0/pan-os-admin/authentication/configure-kerberos-server-authentication.html}{it is trivial to integrate with 
Palo Alto Networks security devices} running PANOS. The user accounts are available on the device after entering the details about
the KDC.

\vspace{2mm}
\ptitle{Container Integration}
\vspace{2mm}

\justifying
Users are able to authenticate against the KDC from a container by simply copying in the ``/etc/krb5.conf'' file and then running ``kinit'' as usual.

\justifying
A full working example Dockerfile has been included for testing and proof of concept. Consider the following execution provided for demonstration purposes.

\begin{mybox}{\thetcbcounter: Authenticating against KDC from a Container}
	\lstinputlisting{code/container-krb}
\end{mybox}

\section{\label{sec:summary}Summary}

\justifying
This project paper demonstrates the configuration and capabilities of Kerberos and OpenLDAP, and outlines the reasoning behind the need for adopting a
centralized mechanism for authentication.

\vspace{2mm}
\ptitle{Future Direction}
\vspace{2mm}

\begin{raggedright}
	\begin{enumerate}
		\item Containerize the KDC and LDAP servers and move the entire ecosystem into Kubernetes.
		\item Integration of Kerberos with NFS and \href{https://sssd.io/}{SSSD}
	\end{enumerate}
\end{raggedright}

% backmatter: appendix, bibliography, index, postface
% \printindex

\clearpage
\begin{versionhistory}
    \vhEntry{v0.1}{March 18th, 2014}{Franklin Diaz}{Initial Draft}
	\vhEntry{v0.2}{Sept. 29th, 2021}{Franklin Diaz}{Updates}
\end{versionhistory}
\nocite{*}
\bibliographystyle{plain}
\bibliography{mybib.bib}
% \include{backmatter/about-author}
\end{document}
