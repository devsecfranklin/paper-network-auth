\section{Resources)
	
	http://www.debian-administration.org/articles/570
	http://techpubs.spinlocksolutions.com/dklar/kerberos.html
	http://mindref.blogspot.com/2010/12/debian-kerberos-slave.html
	http://mindref.blogspot.com/2010/12/kerberos-dns-discovery.html
	
\section{Machines and Packages}

\section{OpenAFS}

OpenLDAP has to have support for SASL and GSSAPI. SASL can use GSSAPI to extend it's auth mechanisms. SASL and GSSAPI
are frameworks that various authentication providers can be plugged into, for example Kerberos or NTLM.

Here's an example to help make this a little clearer (brutally simplified for clarity's sake):

1. Client connects to server and says, "I support SASL! How should I authenticate myself?"
2. Server receives the connection and responds, "I also support SASL, and can use these mechanisms, in descending order of preference: GSSAPI, CRAM-MD5, PLAIN."
3. Client responds, "Of the choices, I'd like to use GSSAPI."
4. Server responds "GSSAPI? Capital. I support Kerberos and NTLM."
5. Client responds "Let's use Kerberos. Here's my encrypted ticket etc. etc."